\documentclass[12pt, a4paper, oneside]{ctexart}
\usepackage{amsmath, amsthm, amssymb, bm, color, framed, graphicx, hyperref, mathrsfs}

\title{\textbf{随机过程作业16}}
\author{周强(119) \ 电子学院 \ \ \  202128019427002}
\date{\today}
\linespread{1.5}
\definecolor{shadecolor}{RGB}{241, 241, 255}
\newcounter{problemname}
\newenvironment{problem}{\begin{shaded}\stepcounter{problemname}\par\noindent\textbf{题目\arabic{problemname}. }}{\end{shaded}\par}
\newenvironment{solution}{\par\noindent\textbf{解答. }}{\par}
\newenvironment{note}{\par\noindent\textbf{题目\arabic{problemname}的注记. }}{\par}

\begin{document}

\maketitle

\begin{problem}
    设有一线性系统, 其输入为零均值白高斯噪声 $n(t)$, 其功率谱密度为 $\frac{N_{0}}{2}$, 系统的冲激 响应为:
$$
h(t)=\left\{\begin{array}{cc}
e^{-\alpha t}, & t \geq 0 \\
0, & t<0
\end{array}\right.
$$
此线性系统的输出为 $\xi(t)$ 。令: $\eta(t)=\xi(t)-\xi(t-T)$, 其中 $T>0$ 为一常数, 试求过 程 $\eta(t)$ 的一维概率密度函数。
\end{problem}

\begin{solution}
    由于线性系统的输入是高斯白噪声,则输出$\xi(t)$是高斯过程,$\eta(t) = \xi(t) - \xi(t-T)$也是高斯过程。此线性系统的转移函数为
    \begin{align*}
        & H(j \omega) = \mathcal{F}  \{ h(t) \} 
        =
        \frac{1}{\alpha + j \omega}
        \\
        & |H(j \omega)|^2 
        = 
        \frac{1}{\alpha ^ 2 + \omega ^ 2}
    \end{align*}
    则$\xi (t)$的功率谱函数为
    \begin{align*}
        S_{\xi}(\omega) 
        =
        |H(j \omega)|^2 S_{n}(t)
        =
        \frac{N_0}{2(\alpha ^ 2 + \omega ^ 2)}
    \end{align*}
    则$\xi(t)$的相关函数为
    \begin{align*}
        R_{\xi}(\tau) = \frac{N_0}{4 \alpha} e ^ {-\alpha |\tau|}
    \end{align*}
\end{solution}

\end{document}