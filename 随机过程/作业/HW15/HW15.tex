\documentclass[12pt, a4paper, oneside]{ctexart}
\usepackage{amsmath, amsthm, amssymb, bm, color, framed, graphicx, hyperref, mathrsfs}

\title{\textbf{随机过程作业15}}
\author{周强(119) \ 电子学院 \ \ \  202128019427002}
\date{\today}
\linespread{1.5}
\definecolor{shadecolor}{RGB}{241, 241, 255}
\newcounter{problemname}
\newenvironment{problem}{\begin{shaded}\stepcounter{problemname}\par\noindent\textbf{题目\arabic{problemname}. }}{\end{shaded}\par}
\newenvironment{solution}{\par\noindent\textbf{解答. }}{\par}
\newenvironment{note}{\par\noindent\textbf{题目\arabic{problemname}的注记. }}{\par}

\begin{document}

\maketitle

\begin{problem}
    设 $\{X(t), t \geq 0\}$ 是一实的零初值正交增量过程, 且 $X(t) \sim N\left(\mu, \sigma^{2} t\right)$ 。令 $Y(t)=2 X(t)-1, t \geq 0$ 。试求过程 $\{Y(t), t \geq 0\}$ 的相关函数 $R_{Y}(s, t)$ 。
\end{problem}

\begin{solution}
    由相关函数的定义可知
    \begin{align*}
        & R_{Y}(s,t)
        =
        E\{
            Y(s)Y(t)
        \}
        =
        E\{
            [2X(s)-1][2X(t)-1]
        \}
        \\
        & =
        E\{
            4X(s)X(t)-2X(s)-2X(t)+1
        \}
    \end{align*}
    由于$\{X(t),t \geq 0 \}$是初值为0的正交增量随机过程,当$0 \leq s < t$时有,
    \begin{align*}
        &
        E\{
            X(s)X(t)  
        \}
        =
        E\{
            [X(s)-X(0)][X(t)-X(0)]  
        \}
        \\
        &=
        E\{
            [X(s)-X(0)][X(t)-X(s)+X(s)-X(0)]  
        \}
        \\
        &=
        E\{
            [X(s)-X(0)][X(s)-X(0)]  
        \}
        \\
        &=
        E\{
            [X(s)]^{2}
        \}
        =
        E\{
            X(s)
        \}^{2}
        +D\{
            X(s)  
        \}
        =\mu ^ {2} + \sigma ^ 2 s
    \end{align*}
    因此
    \begin{align*}
        R_Y(s,t)=4 \mu ^ 2 + 4 \sigma ^ 2 s - 4\mu +1
    \end{align*}
    同理可证,当$0 \leq t < s$时有,
    \begin{align*}
        R_Y(s,t)=4 \mu ^ 2 + 4 \sigma ^ 2 t - 4\mu +1
    \end{align*}
    综上所述,
    \begin{align*}
        R_Y(s,t)=4 \mu ^ 2 + 4 \sigma ^ 2 \min\{s,t\} - 4\mu +1
    \end{align*}
\end{solution}

\begin{problem}
    设有随机过程 $X(t)=2 Z \sin (t+\Theta),-\infty<t<+\infty$, 其中 $Z$、$ \Theta$ 是相互独立的随机 变量, $Z \sim N(0,1), P(\Theta=\pi / 4)=P(\Theta=-\pi / 4)=1 / 2$ 。问过程 $X(t)$ 是否均方可积 过程? 说明理由。
\end{problem}

\begin{solution}
    $X(t)$的均值函数为
    \begin{align*}
        E\{X(t)\}
        =
        E\{2Z\sin(t + \Theta)\}
        =
        2E\{Z\}E\{\sin(t + \Theta)\}
        =0
    \end{align*}
    $X(t)$的相关函数为
    \begin{align*}
        & R_{X}(t,s)
        =
        E\{X(t)X(s)\}
        =
        E\{(2Z\sin(t + \Theta))(2Z\sin(s + \Theta))\}
        \\
        & =
        4E\{Z^2\}
        E\{
            \sin (t+\Theta)
            \sin (s+\Theta)
        \}
        =
        4
        E\{
            \frac{1}{2}[
                \cos (t-s)
                -
                \cos (s+t+2\Theta)
            ]
        \}
        \\
        &=
        2 \cos (t-s)
    \end{align*}
    由此可知,随机过程$X(t)$是平稳过程,且均方可积。
\end{solution}

\begin{problem}
    设随机过程 $\xi(t)=X \cos 2 t+Y \sin 2 t,-\infty<t<+\infty$, 其中随机变量 $X$ 和 $Y$ 独立同分 布。
    \begin{itemize}
        \item [(1)]
        如果 $X \sim U(0,1)$, 问过程 $\xi(t)$ 是否平稳过程?说明理由;
        \item [(2)]
        如果 $X \sim N(0,1)$, 问过程 $\xi(t)$ 是否均方可微? 说明理由。

    
    \end{itemize}

\end{problem}

\begin{solution}
    \begin{itemize}
        \item [(1)] $\xi(t)$的均值函数为
        \begin{align*}
            &
            E\{
                \xi (t)    
            \}
            =
            E\{
                X \cos(2t) + Y \sin(2t)    
            \}
            \\
            &=
            E\{
                X
            \}  
            \cos(2t)
            +
            E\{
                Y
            \}  
            \sin(2t)
            \\
            &=
            \frac{1}{2}
            \left(
                \cos(2t) + \sin(2t)
            \right) 
        \end{align*}
        均值函数不是常数,因此$\xi(t)$不是平稳过程。
        \item [(2)]
        因为$X$和$Y$是独立同分布的随机变量,且$X \sim N(0,1)$,则
        \begin{align*}
            E\{X^2\}=E\{Y^2\}=1,
            E\{XY\}=0
        \end{align*}
        $\xi(t)$的相关函数为
        \begin{align*}
            R_{\xi}(s,t)
            &=
            E\{
               (X \cos(2t) + Y \sin(2t))
               (X \cos(2s) + Y \sin(2s)) 
            \}
            \\
            &=
            E\{
                X^2 
            \}
            (\cos(2t)\cos(2s))
            \\
            &
            +
            E\{
                XY 
            \}
            (\cos(2t)\sin(2s)+\cos(2s)\sin(2t))
            \\
            &
            +
            E\{
                Y^2 
            \}
            (\sin(2t)\sin(2s))
            \\
            &=
            \cos(2t)\cos(2s)
            +
            \sin(2t)\sin(2s)
            \\&=
            \cos(2t-2s)
        \end{align*}
    因此,$\xi(t)$是平稳过程,且均方可微。
    \end{itemize}
\end{solution}

\begin{problem}
    设随机过程 $\{X(t) ;-\infty<t<+\infty\}$ 是一实正交增量过程, 并且 $E\{X(t)\}=0$, 及满足:
    $$
    E\left\{[X(t)-X(s)]^{2}\right\}=|t-s|, \quad-\infty<s, t<+\infty \text {; }
    $$
    令: $Y(t)=X(t)-X(t-1),-\infty<t<+\infty$, 试证明 $Y(t)$ 是平稳过程。
\end{problem}

\begin{solution}
    由题意易知,$Y(t)$的均值函数为$E\{Y(t)\}=0$。$Y(t)$的相关函数为
    \begin{align*}
        R_{Y}(t,s)
        =
        E\{Y(t)Y(s)\}
        =
        E\{
            [X(t)-X(t-1)]
            [X(s)-X(s-1)]
        \}
    \end{align*}
    不失一般性地假设$s>t$。当$s>t+1$时,由$X(t)$的增量正交性可知,
    \begin{align*}
        R_{Y}(t,s)
        =0
    \end{align*}
    当$t<s<t+1$时,
    \begin{align*}
        &
        R_{Y}(t,s)
        =
        E\{
            [X(t)-X(t-1)]
            [X(s)-X(t)+X(t)-X(s-1)]
        \}
        \\
        & =
        E\{
            [X(t)-X(t-1)]
            [X(t)-X(s-1)]
        \}
        \\
        &=
        E\{
            [X(t)-X(s-1)+X(s-1)-X(t-1)]
            [X(t)-X(s-1)]
        \}
        \\
        &=
        E\{
            [X(t)-X(s-1)]^2
        \}
        =|t-s+1|
    \end{align*}
    则$Y(t)$的相关函数仅与时间差有关。同理可证$t>s$的情况。综上,$Y(t)$是平稳过程。
\end{solution}

\begin{problem}
    设 $\xi(t)=X \sin (Y t) ; t \geq 0$, 而随机变量 $X 、 Y$ 是相互独立且都服从 $[0,1]$ 上的均匀分布, 试求此过程的均值函数及相关函数。并问此过程是否是平稳过程, 是否连续、可导?
\end{problem}
\begin{solution}
    由题意知,$\xi (t)$的均值函数和相关函数为
    \begin{align*}
        E\{\xi (t)\}
        =
        E\{X\}E\{\sin(Yt)\}
        =
        \frac{1}{2}
        \int_{0}^{1}{
            \sin(yt)dy
        }
        =
        \frac{1-\cos t}{2t}
    \end{align*}
    \begin{align*}
        R_{\xi}(t,s)
        & =
        E\{X^2\}
        E\{
            \sin(Yt)
            \sin(Ys)
        \}
        \\
        & =
        \frac{1}{3}
        E\{
            \frac{1}{2}
            [
                \cos(Y(t-s))
                -
                \cos(Y(t+s))
            ]    
        \}
        \\
        &=
        \frac{1}{6}
        \left[
            \frac{\sin(t-s)}{t-s}
            -
            \frac{\sin(t+s)}{t+s}
        \right]
    \end{align*}
\end{solution}

\begin{problem}
设 $\{X(t),t\in R\}$ 是连续平稳过程, 均值为m, 协方差函数为 $C_{X}(\tau)=a e^{-b|\tau|}$, 其中$\tau \in R, a, b>0$ 。对固定的 $T>0$, 令 $Y=T^{-1} \int_{0}^{T} X(s) d s$, 证明: $E\{Y\}=m$, $\operatorname{Var}(Y)=2 a\left[(b T)^{-1}-(b T)^{-2}\left(1-e^{-b T}\right)\right]$ 。
\end{problem}

\begin{solution}
    $Y$的均值函数为
    \begin{align*}
        E\{Y\}
        =
        E\{
            T^{-1} \int_{0}^{T} X(s) d s
        \}
        =
        T^{-1} 
        \int_{0}^{T} E\{X(s)\} ds
        =
        m
    \end{align*}
    $Y$的二阶矩为
    \begin{align*}
        E\{Y^2\}
        & =
        E\left\{
            T^{-2}
            \left(
                \int_{0}^{T} X(s) ds
            \right)
            ^2  
        \right\}
        \\
        &=
        T^{-2}
        E\left\{
            \left(
                \int_{0}^{T} X(s) ds
            \right)
            \left(
                \int_{0}^{T} X(u) du
            \right)
        \right\}
        \\
        &=
        T^{-2}
        E\left\{
                \int_{0}^{T} {
                    \int_{0}^{T}
                                {
                                    X(u)
                                    X(s)
                                    du
                                    ds
                                }
                } 
        \right\}
        \\
        &=
        T^{-2}
        \int_{0}^{T} {
            \int_{0}^{T}
                        E\{
                            X(u)
                            X(s)
                        \}
                            du
                            ds
        } 
        \\
        &=
        T^{-2}
        \int_{0}^{T} {
            \int_{0}^{T}
            R_X(u-s)
            duds
        }
        \\
        &=
        T^{-2}
        \int_{0}^{T} {
            \int_{0}^{T}
            C_X(u-s) + m^2
            duds
        }
        \\
        &=
        2 a\left[(b T)^{-1}-(b T)^{-2}\left(1-e^{-b T}\right)\right] 
        + m^2
    \end{align*}
    则Y的方差为
    \begin{align*}
        Var\{Y\}
        =
        E\{Y^2\}
        -(E\{Y\})^2
        =
        2 a\left[(b T)^{-1}-(b T)^{-2}\left(1-e^{-b T}\right)\right] 
    \end{align*}

\end{solution}

\begin{problem}
    设 $(X, Y) \sim N\left(0,0, \sigma_{1}^{2}, \sigma_{2}^{2}, \rho\right)$ , 令 $X(t)=X+t Y$ , 以 及 $Y(t)=\int_{0}^{t} X(u) d u$ , $Z(t)=\int_{0}^{t} X^{2}(u) d u$, 对于任意 $0 \leq s \leq t$,
    \begin{itemize}
        \item [(1)]  求 $E\{X(t)\}, E\{Y(t)\}, E\{Z(t)\}, \operatorname{Cov}(X(s), X(t)), \operatorname{Cov}(Y(s), Y(t))$;
        \item [(2)] 证明 $X(t)$ 在 $t>0$ 上均方连续、均方可导;
        \item [(3)] 求 $Y(t)$ 及 $Z(t)$ 的均方导数。

    \end{itemize}

\end{problem}

\begin{solution}
    \begin{itemize}
        \item [(1)]
        因为$(X, Y) \sim N\left(0,0, \sigma_{1}^{2}, \sigma_{2}^{2}, \rho\right)$,则
        \begin{align*}
            E\{X(t)\}
            & =
            E\{X\} + t E\{Y\}
            = 0
            \\
            E\{Y(t)\}
            & =
            E\left\{
                \int_{0}^{t} X(u) d u
            \right\}
            =
            \int_{0}^{t} E\{X(u)\} d u
            =0
        \end{align*}
        \item [(2)]
        

        \item [(3)]
    
    \end{itemize}
\end{solution}

\begin{problem}
    设随机过程 $\{X(t) ;-\infty<t<+\infty\}$ 是均值为零、自相关函数为 $R_{X}(\tau)$ 的实平稳正 态过程。设 $X(t)$ 通过线性全波检波器后, 其输出为 $Y(t)=|X(t)|$, 试求:
    \begin{itemize}
        \item [(1)] 随机过程 $Y(t)$ 的相关函数 $R_{Y}(\tau)$, 并说明其是否为平稳过程;
        \item [(2)] 随机过程 $Y(t)$ 的均值和方差;
        \item [(3)] 随机过程 $Y(t)$ 的一维概率分布密度函数 $f_{Y}(y)$ 。
    \end{itemize}

\end{problem}

\begin{solution}

\end{solution}
\end{document}