\documentclass[12pt, a4paper, oneside]{ctexart}
\usepackage{amsmath, amsthm, amssymb, bm, color, framed, graphicx, hyperref, mathrsfs}

\title{\textbf{随机过程作业15}}
\author{周强(119) \ 电子学院 \ \ \  202128019427002}
\date{\today}
\linespread{1.5}
\definecolor{shadecolor}{RGB}{241, 241, 255}
\newcounter{problemname}
\newenvironment{problem}{\begin{shaded}\stepcounter{problemname}\par\noindent\textbf{题目\arabic{problemname}. }}{\end{shaded}\par}
\newenvironment{solution}{\par\noindent\textbf{解答. }}{\par}
\newenvironment{note}{\par\noindent\textbf{题目\arabic{problemname}的注记. }}{\par}

\begin{document}

\maketitle

\begin{problem}
\end{problem}

\begin{solution}
    由相关函数的定义可知
    \begin{align*}
        & R_{Y}(s,t)
        =
        E\{
            Y(s)Y(t)
        \}
        =
        E\{
            [2X(s)-1][2X(t)-1]
        \}
        \\
        & =
        E\{
            4X(s)X(t)-2X(s)-2X(t)+1
        \}
    \end{align*}
    由于$\{X(t),t \geq 0 \}$是初值为0的正交增量随机过程,当$0 \leq s < t$时有,
    \begin{align*}
        &
        E\{
            X(s)X(t)  
        \}
        =
        E\{
            [X(s)-X(0)][X(t)-X(0)]  
        \}
        \\
        &=
        E\{
            [X(s)-X(0)][X(t)-X(s)+X(s)-X(0)]  
        \}
        \\
        &=
        E\{
            [X(s)-X(0)][X(s)-X(0)]  
        \}
        \\
        &=
        E\{
            [X(s)]^{2}
        \}
        =
        E\{
            X(s)
        \}^{2}
        +D\{
            X(s)  
        \}
        =\mu ^ {2} + \sigma ^ 2 s
    \end{align*}
    因此
    \begin{align*}
        R_Y(s,t)=4 \mu ^ 2 + 4 \sigma ^ 2 s - 4\mu +1
    \end{align*}
    同理可证,当$0 \leq t < s$时有,
    \begin{align*}
        R_Y(s,t)=4 \mu ^ 2 + 4 \sigma ^ 2 t - 4\mu +1
    \end{align*}
    综上所述,
    \begin{align*}
        R_Y(s,t)=4 \mu ^ 2 + 4 \sigma ^ 2 \min\{s,t\} - 4\mu +1
    \end{align*}
\end{solution}

\begin{problem}
    
\end{problem}

\begin{solution}
    
\end{solution}

\begin{problem}
    
\end{problem}

\begin{solution}
    \begin{itemize}
        \item [(1)] $\xi(t)$的均值函数为
        \begin{align*}
            &
            E\{
                \xi (t)    
            \}
            =
            E\{
                X \cos(2t) + Y \sin(2t)    
            \}
            \\
            &=
            E\{
                X
            \}  
            \cos(2t)
            +
            E\{
                Y
            \}  
            \sin(2t)
            \\
            &=
            \frac{1}{2}
            \left(
                \cos(2t) + \sin(2t)
            \right) 
        \end{align*}
        均值函数不是常数,因此$\xi(t)$不是平稳过程。
        \item [(2)]
        因为$X$和$Y$是独立同分布的随机变量,且$X \sim N(0,1)$,则
        \begin{align*}
            E\{X^2\}=E\{Y^2\}=1,
            E\{XY\}=0
        \end{align*}
        $\xi(t)$的相关函数为
        \begin{align*}
            R_{\xi}(s,t)
            &=
            E\{
               (X \cos(2t) + Y \sin(2t))
               (X \cos(2s) + Y \sin(2s)) 
            \}
            \\
            &=
            E\{
                X^2 
            \}
            (\cos(2t)\cos(2s))
            \\
            &
            +
            E\{
                XY 
            \}
            (\cos(2t)\sin(2s)+\cos(2s)\sin(2t))
            \\
            &
            +
            E\{
                Y^2 
            \}
            (\sin(2t)\sin(2s))
            \\
            &=
            \cos(2t)\cos(2s)
            +
            \sin(2t)\sin(2s)
            \\&=
            \cos(2t-2s)
        \end{align*}
    因此,$\xi(t)$是平稳过程,且均方可微。
    \end{itemize}
\end{solution}
\end{document}