\documentclass[12pt, a4paper, oneside]{ctexart}
\usepackage{amsmath, amsthm, amssymb, bm, color, framed, graphicx, hyperref, mathrsfs}

\title{\textbf{随机过程作业15}}
\author{周强(119) \ 电子学院 \ \ \  202128019427002}
\date{\today}
\linespread{1.5}
\definecolor{shadecolor}{RGB}{241, 241, 255}
\newcounter{problemname}
\newenvironment{problem}{\begin{shaded}\stepcounter{problemname}\par\noindent\textbf{题目\arabic{problemname}. }}{\end{shaded}\par}
\newenvironment{solution}{\par\noindent\textbf{解答. }}{\par}
\newenvironment{note}{\par\noindent\textbf{题目\arabic{problemname}的注记. }}{\par}

\begin{document}

\maketitle

\begin{problem}
    设 $\{X(t), t \geq 0\}$ 是一实的零初值正交增量过程, 且 $X(t) \sim N\left(\mu, \sigma^{2} t\right)$ 。令 $Y(t)=2 X(t)-1, t \geq 0$ 。试求过程 $\{Y(t), t \geq 0\}$ 的相关函数 $R_{Y}(s, t)$ 。
\end{problem}

\begin{solution}
    由相关函数的定义可知
    \begin{align*}
        & R_{Y}(s,t)
        =
        E\{
            Y(s)Y(t)
        \}
        =
        E\{
            [2X(s)-1][2X(t)-1]
        \}
        \\
        & =
        E\{
            4X(s)X(t)-2X(s)-2X(t)+1
        \}
    \end{align*}
    由于$\{X(t),t \geq 0 \}$是初值为0的正交增量随机过程,当$0 \leq s < t$时有,
    \begin{align*}
        &
        E\{
            X(s)X(t)  
        \}
        =
        E\{
            [X(s)-X(0)][X(t)-X(0)]  
        \}
        \\
        &=
        E\{
            [X(s)-X(0)][X(t)-X(s)+X(s)-X(0)]  
        \}
        \\
        &=
        E\{
            [X(s)-X(0)][X(s)-X(0)]  
        \}
        \\
        &=
        E\{
            [X(s)]^{2}
        \}
        =
        E\{
            X(s)
        \}^{2}
        +D\{
            X(s)  
        \}
        =\mu ^ {2} + \sigma ^ 2 s
    \end{align*}
    因此
    \begin{align*}
        R_Y(s,t)=4 \mu ^ 2 + 4 \sigma ^ 2 s - 4\mu +1
    \end{align*}
    同理可证,当$0 \leq t < s$时有,
    \begin{align*}
        R_Y(s,t)=4 \mu ^ 2 + 4 \sigma ^ 2 t - 4\mu +1
    \end{align*}
    综上所述,
    \begin{align*}
        R_Y(s,t)=4 \mu ^ 2 + 4 \sigma ^ 2 \min\{s,t\} - 4\mu +1
    \end{align*}
\end{solution}

\begin{problem}
    设有随机过程 $X(t)=2 Z \sin (t+\Theta),-\infty<t<+\infty$, 其中 $Z$、$ \Theta$ 是相互独立的随机 变量, $Z \sim N(0,1), P(\Theta=\pi / 4)=P(\Theta=-\pi / 4)=1 / 2$ 。问过程 $X(t)$ 是否均方可积 过程? 说明理由。
\end{problem}

\begin{solution}
    
\end{solution}

\begin{problem}
    设随机过程 $\xi(t)=X \cos 2 t+Y \sin 2 t,-\infty<t<+\infty$, 其中随机变量 $X$ 和 $Y$ 独立同分 布。
    \begin{itemize}
        \item [(1)]
        如果 $X \sim U(0,1)$, 问过程 $\xi(t)$ 是否平稳过程?说明理由;
        \item [(2)]
        如果 $X \sim N(0,1)$, 问过程 $\xi(t)$ 是否均方可微? 说明理由。

    
    \end{itemize}

\end{problem}

\begin{solution}
    \begin{itemize}
        \item [(1)] $\xi(t)$的均值函数为
        \begin{align*}
            &
            E\{
                \xi (t)    
            \}
            =
            E\{
                X \cos(2t) + Y \sin(2t)    
            \}
            \\
            &=
            E\{
                X
            \}  
            \cos(2t)
            +
            E\{
                Y
            \}  
            \sin(2t)
            \\
            &=
            \frac{1}{2}
            \left(
                \cos(2t) + \sin(2t)
            \right) 
        \end{align*}
        均值函数不是常数,因此$\xi(t)$不是平稳过程。
        \item [(2)]
        因为$X$和$Y$是独立同分布的随机变量,且$X \sim N(0,1)$,则
        \begin{align*}
            E\{X^2\}=E\{Y^2\}=1,
            E\{XY\}=0
        \end{align*}
        $\xi(t)$的相关函数为
        \begin{align*}
            R_{\xi}(s,t)
            &=
            E\{
               (X \cos(2t) + Y \sin(2t))
               (X \cos(2s) + Y \sin(2s)) 
            \}
            \\
            &=
            E\{
                X^2 
            \}
            (\cos(2t)\cos(2s))
            \\
            &
            +
            E\{
                XY 
            \}
            (\cos(2t)\sin(2s)+\cos(2s)\sin(2t))
            \\
            &
            +
            E\{
                Y^2 
            \}
            (\sin(2t)\sin(2s))
            \\
            &=
            \cos(2t)\cos(2s)
            +
            \sin(2t)\sin(2s)
            \\&=
            \cos(2t-2s)
        \end{align*}
    因此,$\xi(t)$是平稳过程,且均方可微。
    \end{itemize}
\end{solution}

\begin{problem}
    设随机过程 $\{X(t) ;-\infty<t<+\infty\}$ 是一实正交增量过程, 并且 $E\{X(t)\}=0$, 及满足:
    $$
    E\left\{[X(t)-X(s)]^{2}\right\}=|t-s|, \quad-\infty<s, t<+\infty \text {; }
    $$
    令: $Y(t)=X(t)-X(t-1),-\infty<t<+\infty$, 试证明 $Y(t)$ 是平稳过程。
\end{problem}

\begin{solution}

\end{solution}

\begin{problem}
    设 $\xi(t)=X \sin (Y t) ; t \geq 0$, 而随机变量 $X 、 Y$ 是相互独立且都服从 $[0,1]$ 上的均匀分布, 试求此过程的均值函数及相关函数。并问此过程是否是平稳过程, 是否连续、可导?
\end{problem}
\begin{solution}

\end{solution}

\begin{problem}
设 $\{X(t),t\in R\}$ 是连续平稳过程, 均值为m, 协方差函数为 $C_{X}(\tau)=a e^{-b|\tau|}$, 其中$\tau \in R, a, b>0$ 。对固定的 $T>0$, 令 $Y=T^{-1} \int_{0}^{T} X(s) d s$, 证明: $E\{Y\}=m$, $\operatorname{Var}(Y)=2 a\left[(b T)^{-1}-(b T)^{-2}\left(1-e^{-b T}\right)\right]$ 。
\end{problem}

\begin{solution}

\end{solution}
\end{document}